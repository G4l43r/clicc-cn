\chapter{Looking Inside Files (grep, select-string)}

\section{Do This}

I'm showing you a little trick for typing text into a file real quick.  If you
do \verb|cat > somefile.txt| then \program{cat} will read whatever you type and
then write it to that file.  The improtant thing though is that you have to
"close" the file by typing CTRL-d (CTRL-z on Windows).  If you can't figure this 
out just use a text editor to make the \file{newfile.txt} and \file{oldfile.txt}.

\begin{code}{Linux/OSX Exercise 18}
<< d['code/ex18.sh-session|pyg|l'] >>
\end{code}

\begin{code}{Windows Exercise 18}
<< d['code/ex18-win.sh-session|pyg|l'] >>
\end{code}

\section{You Learned This}

You made two files that were almost the same, except one had "new" and the other 
had "old" in it.  Then you searched for different words in those files.  You should
see how you can look for words, and actually you can find whole sentences in
files if you put them in quotes.

\section{Do More}

\begin{enumerate}
\item Use quotes to find "new file" and "old file" and "This is".
\item Take the list of videos you created (or any other list) and use it to find some videos you want to find.
\end{enumerate}

