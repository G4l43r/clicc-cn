\chapter{Looking Inside Files (grep, select-string)}

\section{Do This}

I'm showing you a little trick for typing text into a file very quickly.  If
you do \verb|cat > somefile.txt| then \program{cat} will read whatever you type
and then write it to that file.  \emph{On Windows} it's done with \verb|echo > somefile.txt|.
The important thing, though, is that you have to "close" the
file by typing CTRL-d.

If you can't figure this out just use a text editor to make the
\file{newfile.txt} and \file{oldfile.txt}.

\begin{code}{Linux/Mac OSX Exercise 18}
<< d['code/ex18.sh-session|pyg|l'] >>
\end{code}

In Windows a similar trick is to do \verb|echo > somefile.txt| and then you'll
be prompted for each line.  Give an empty line to stop entering text.

\begin{code}{Windows Exercise 18}
<< d['code/ex18-win.sh-session|pyg|l'] >>
\end{code}

\section{You Learned This}

You made two files that were almost the same, except one had "new" and the other 
had "old" in it.  Then you searched for different words in those files.  You can look for words, but you can also find whole sentences in
files if you put them in quotes.

\section{Do More}

\begin{enumerate}
\item Use quotes to find "new file" and "old file" and "This is".
\item Take the list of videos you created (or any other list) and use it to find some videos you want to find.
\item Unix: You can use \verb|-i| to ignore case with \program{grep}.  Try \verb|grep -i new *.txt|
\end{enumerate}

