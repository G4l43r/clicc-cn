\chapter{Copy A File (cp, ROBOCOPY)}

\section{Do This}

\begin{code}{Linux/OSX Exercise 10}
\begin{Verbatim}
<< d['code/ex10.sh-session'] >>
\end{Verbatim}
\end{code}

\begin{code}{Windows Exercise 10}
\begin{Verbatim}
<< d['code/ex10-win.sh-session'] >>
\end{Verbatim}
\end{code}

\section{You Learned This}

Now you can copy files.  It's simple to just take a file and copy it to a new
one.  In this exercise I also make a new directory and copy a file into that
directory.

I'm going to tell you a secret about programmers and system administrators.
They are lazy.  I'm lazy.  My friends are lazy.  That's why we use computers.
We like to make computers do boring things for us.  In the exercises so far
you have been typing repetitive boring commands so that you can learn them,
but usually it's not like this.  Usually if you find yourself doing something
boring and repetitive there's probably a programmer who has figured out
how to make it easier, you just don't know about it.

The other thing about programmers is they aren't nearly as clever as you
think.  If you over think what to type then you'll probably get it wrong.
Instead, try to imagine what the name of a command is to you and try it.
Chances are that's it's name or some abbreviation.  After you can't figure
it out intuitively then go ask around, and hopefully it's not something
really stupid like ROBOCOPY.


\section{Do More}

\begin{enumerate}
\item Use the \program{cp -r} command or \program{ROBOCOPY} to copy a
    whole.
\item Copy a file to your home directory or desktop.
\item Find these files in your graphical user interface and open them
    in a text editor.
\end{enumerate}

