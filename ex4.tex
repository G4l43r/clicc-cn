\chapter{Make A Directory (mkdir)}

\section{Do This}

\begin{code}{Linux/OSX Exercise 4}
<< d['code/ex4.sh-session|pyg|l'] >>
\end{code}

\begin{code}{Windows Exercise 4}
<< d['code/ex4-win.sh-session|pyg|l'] >>
\end{code}

\section{You Learned This}

Now we get into typing more than one command.  These are all the different ways
you can run mkdir.  What's mkdir do?  It make directories.  Why are you asking
that?  You should be doing your index cards and getting your commands memorized.
If you don't know that "mkdir makes directories" then keep working the index
cards.

What does it mean to make a directory?  You might call directories "folders".  They're
the same thing.  All you did above is create directories, that are inside directories,
that are then again inside others.  This is called a "path" and it's a way of saying
"first temp, then stuff, then things and that's where I want it".  It's kind of a
set of directions to the computer of where you want to put something in the tree
of folders (directories) that make up your computer's hard disk.

\section{Do More}

\begin{enumerate}
\item The concept of a "path" might confuse you at this point.  Don't worry we'll 
    do a lot more with them and then you'll get it.
\item Make 20 other directories inside the temp directory in various levels.  Go look at it
    with a graphical file browser.
\end{enumerate}

