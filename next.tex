\chapter{Next Steps}

You have completed the crash course.  At this point you should be a 
barely capable shell user.  There's a whole huge list of tricks and key
sequences you don't know yet, and I'm going to give you a few final
places to go research more.

\section{Unix Bash References}

The shell you've been using is called Bash.  It's not the greatest shell
but it's everywhere and has a lot of features so it's a good start.  Here's a 
short list of links about Bash you should go read:

\begin{description}
\item[Bash Cheat Sheet] \href{http://cli.learncodethehardway.org/bash_cheat_sheet.pdf} created by \href{http://freeworld.posterous.com/65140847}{Raphaël} and CC
licensed.
\item[Reference Manual] \href{http://www.gnu.org/software/bash/manual/bashref.html}
\end{description}


\section{PowerShell References}

On Windows there's really only PowerShell.  Here's a list of useful links for you
related to PowerShell:

\begin{description}
\item[Owner's Manual] \href{http://technet.microsoft.com/en-us/library/ee221100.aspx}
\item[Cheat Sheet] \href{http://www.microsoft.com/download/en/details.aspx?displaylang=en&id=7097}
\end{description}

\section{Go Forth}

From now on you shouldn't be afraid of using the command line.  If you are
aspiring to be a programmer, it is the best first step to understand how
a computer is a "language machine".  The shell is much like a tiny little
programming language that most people can get easily.

My advice for getting good at the the CLI is to force yourself to use it every
day, no matter how painful it seems.  Part of the "pain" is having to remember
the commands without a visual cue.  The advantage of a GUI is that you get a
cue from the graphics to remind you of how to use the tool.  With the CLI you
have to dredge every command from nothing, which is irritating at first.

However, I have a trick for you to get over this pain.  Create your own cheat
sheet of CLI tricks you use all the time.  Make yourself use the CLI to do
things, and when you run into something you just \emph{have} to use again, 
write on your own cheat sheet.  The next time you need to do that, look at
your cheat sheet and you'll remember.

Eventually you won't need the cheat sheet.  In fact, I'd say most of my
daily shell usage consists of 10 commands, most of which are in this little
book.  Memorizing 10 commands is really easy, so there's nothing stopping you.

If you run into trouble with this book, feel free to email me at 
help@learncodethehardway.org and I'll help out.
