\chapter{List Directory (ls)}

\section{Do This}

\begin{code}{Linux/OSX Exercise 6}
\begin{Verbatim}
<< d['code/ex6.sh-session'] >>
\end{Verbatim}
\end{code}

\begin{code}{Windows Exercise 6}
\begin{Verbatim}
<< d['code/ex6-win.sh-session'] >>
\end{Verbatim}
\end{code}

\section{You Learned This}

The \program{ls} command just lists out the contents of the directory you
are currently in.  You can see me use \program{cd} to change into different
directories and then list what's in them, which is the next directory I
go into.

There's a lot of options to the ls command, but you'll learn how to get
help on those later when we cover the \program{help} command.

\section{Do More}

\begin{enumerate}
\item On unix try the \verb|ls -lR| command while you're in \file{temp}.
\item On Windows do the same thing with TODO(what?).
\item Use \program{cd} to get to other directories on your computer then use \program{ls} to see what's in them.
\item Update your notebook with new questions.  I know you have some because I'm
    not covering everything about this command.
\end{enumerate}

