\chapter{Paths, Folders, Directories (pwd)}

\section{Do This}

I'm going to tell you how to read these "sessions" that I show you.  You don't
type everything I list here, just some of the parts:

\begin{enumerate} 
\item You do \emph{not} type in the \verb|$| (unix) or
    \verb|>| (Windows).  That's just me showing you my session so you can see what
    I got.
\item You type in the stuff after this, then hit enter.  So if I have \verb|$ pwd| you type just \verb|pwd| and hit enter.
\item You can then see what I got for output followed by another \verb|$| or \verb|>| prompt.  That stuff is the output and you should see the same thing.
\end{enumerate}

Let's do a simple first command so you can get the hang of this:

\begin{code}{Linux/OSX Exercise 2}
<< d['code/ex2.sh-session|pyg|l'] >>
\end{code}

\begin{code}{Windows Exercise 2}
<< d['code/ex2-win.sh-session|pyg|l'] >>
\end{code}

\section{You Learned This}

First up, your prompt will look different from mine.  You may have your user
name before the \verb|$| and the name of your computer.  On Windows it will
probably look different too.  The key is that you see the pattern of:

\begin{enumerate}
\item There's a prompt.
\item You type a command there, in this case pwd.
\item It printed something.
\item Repeat.
\end{enumerate}

You just learned what \program{pwd} does, which means "print working
directory".  What's a directory?  It's a folder.  Folder and directory are the
same thing, and used interchangeably.  When you open your file browser on your
computer to graphically find files, you are walking through folders.  Those
folders are the exact same things as these "directories" we're going to work
with.

\section{Do More}

\begin{enumerate}
\item Type \program{pwd} 20 times and each time say "print working directory".
\item Write down the path that this command gives you.  Find it with your 
    graphical file browser of choice.
\item No, seriously, type it 20 times and say it out loud.  Sssh.  Just do it.
\end{enumerate}

