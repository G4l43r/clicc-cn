\chapter{View A File (less, MORE)}

To do this exercise you're going to do some work using the commands you know so far.
You'll also need a text editor that can make plain text (.txt) files.  Here's
what you do:

\begin{enumerate}
\item Open your text editor and type some stuff into a new file. On OSX this could be TextEdit.  On Windows this might be Notepad.  On Linux this could be GEdit.  Any editor will work.
\item Save that file to your Desktop and name it \file{ex12.txt}.
\item In your shell use the commands you know to \emph{copy} this file
    to your \file{temp} directory that you've been working with.
\end{enumerate}

Once you've done that, complete this exercise.

\section{Do This}

\begin{code}{Linux/Mac OSX Exercise 12}
<< d['code/ex12.sh-session|pyg|l'] >>
\end{code}

That's it. To get out of \program{less} just type \verb|q| (as in quit).

\begin{code}{Windows Exercise 12}
<< d['code/ex12-win.sh-session|pyg|l'] >>
\end{code}

\section{You Learned This}

This is one way to look at the contents of a file.  It's useful because, if the
file is has many lines, it will "page" so that only one screenful at a time
is visible.  In the "Do More" section you'll play with this some more.


\section{Do More}

\begin{enumerate}
\item Open your text file again and repeatedly copy-paste the text so that it's about 50-100 lines long.
\item Copy it to your \file{temp} directory again so you can look at it.
\item Now do the exercise again, but this time page through it.  On Unix you use
    the spacebar and \verb|w| (the letter w) to go down and up.  Arrow keys also work. On Windows just hit ENTER.
\item Look at some of the empty files you created too.
\end{enumerate}

