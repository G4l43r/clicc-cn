\chapter{View A File (less, MORE)}

To do this exercise you're going to do some work using the commands you know so far.
You'll also need a text editor that can make plain text (.txt) files.  Here's
what you do:

\begin{enumerate}
\item Open your text editor and type some stuff into a new file.
\item Save that file to your Desktop and name it \file{ex12.txt}.
\item In your shell use the commands you know to \emph{copy} this file
    to your \file{temp} directory that you've been working with.
\end{enumerate}

Once you've done that, complete this exercise.

\section{Do This}

\begin{code}{Linux/OSX Exercise 12}
\begin{Verbatim}
<< d['code/ex12.sh-session'] >>
\end{Verbatim}
\end{code}

That's it, to get out of \program{less} just type \verb|q| (as in quit).

\begin{code}{Windows Exercise 12}
\begin{Verbatim}
<< d['code/ex12-win.sh-session'] >>
\end{Verbatim}
\end{code}

\section{You Learned This}

This is one way to look at the contents of a file.  It's useful because, if the
file is has many lines it will "page" so that only one screen full at a time
is visible.  In the "Do More" section you'll play with this some more.


\section{Do More}

\begin{enumerate}
\item Open your text file again copy paste the text so that it's about 50-100 lines long.
\item Copy it to your \file{temp} directory again so you can look at it.
\item Now do the exercise again, but this time, page through it.  On Unix you use
    the spacebar and \verb|w| (the letter w) to go down and up.  Arrow keys also work.  On Windows you use (TODO ?).
\end{enumerate}

