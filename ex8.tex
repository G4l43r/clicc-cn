\chapter{Moving Around (pushd, popd)}

\section{Do This}

\begin{code}{Linux/Mac OSX Exercise 8}
<< d['code/ex8.sh-session|pyg|l'] >>
\end{code}

\begin{code}{Windows Exercise 8}
<< d['code/ex8-win.sh-session|pyg|l'] >>
\end{code}

\section{You Learned This}

You're getting into programmer territory with these commands, but they're so
handy I have to teach them to you.  These commands let you temporarily go
to a different directory and then come back to easily switch between the
two.

The \program{pushd} command takes your current directory and "pushes" it
into a list for later, then it \emph{changes} to another directory.  It's
like saying, "Save where I am, then go here."

The \program{popd} command takes the last directory you pushed and "pops"
it off, taking you back there.

Finally, \program{pushd}, if you run it by itself with no arguments, will
switch between your current directory and the last one you pushed.  It's
an easy way to switch between two directories.

\section{Do More}

\begin{enumerate}
\item Use these commands to move around directories all over your
    computer.
\item Remove the \file{i/like/icecream} directories and make your own, then
    move around in them.
\item Explain to yourself the output that \program{pushd} and \program{popd}
    print out to you.  Notice how it works like a stack?
\end{enumerate}

