\chapter{创建目录(mkdir)}

\section{任务}

\begin{code}{Linux/Mac OSX 练习 4}
<< d['code/ex4.sh-session|pyg|l'] >>
\end{code}

\begin{code}{Windows 练习 4}
<< d['code/ex4-win.sh-session|pyg|l'] >>
\end{code}

\section{知识点}

现在我们输入了一条以上的命令。这些命令是使用 mkdir 的不同方法。mkdir 的功能是
什么呢?它是用来 make directory(创建目录)的。你不该问这个问题吧?你应该已经
通过索引卡记住这些了才对。如果你不知道这条,那就说明你要继续在索引卡上下功夫。

创建目录是什么意思?目录又可以叫作“文件夹”,它们是同样的东西。你上面所做的是在
逐层深入地创建目录,目录有时又叫做路径,这里相当于是说“先到 temp,再到 stuff,然后
到 things,这就是我要到的地方。”这是对计算机发出的一系列指向,告诉计算机你想要
把某个东西放到计算机硬盘的某个文件夹(路径)里。


\begin{aside}{Windows, 斜杠和反斜杠}
本书中我使用斜杠 \verb|/| (slash)来表示路径,因为所有的计算机都是这么做的。不过,
Windows 用户应该知道,反斜杠 \verb|\| (backslash)也可以实现同样的功能,别的
Windows 用户可能认为这才是期望的用法。
\end{aside}

\section{更多任务}

\begin{enumerate}
\item 你可能觉得路径的概念还是有些绕人。别担心,我们会做大量的练习让你深入理解。
\item 在 temp 下面再创建 20 个别的目录,深度可以各不相同,然后用文件浏览器检查
你创建的目录。
\item 创建一个名称包含空格的目录,方法是为名称添加一个引号:\verb|mkdir "I Have Fun"|
\item 如果 \file{temp} 目录已经存在,你要创建它时将会得到一条错误信息。使用 
\program{cd} 转到一个别的目录下面试试创建 temp 目录,如果是 Windows 的话,\file{Desktop}
是个不错的选择。
\end{enumerate}

