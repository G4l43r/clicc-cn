\chapter{文件查找内容 (grep, select-string)}

\section{任务}

我来教你一个快速将文本写入一个文件的方法。如果你执行 \verb|cat > somefile.txt|,
\program{cat} 将会把你接下来键入的所有东西都写入这个文件中,Windows 下你需要
执行 \verb|echo > somefile.txt|。很重要的一点是,你需要用 CTRL-d 来结束你的输入。

如果你现在学不会这个操作,你可以使用文本编辑器来创建 \file{newfile.txt} 和 
\file{oldfile.txt}。

\begin{code}{Linux/Mac OSX 练习 18}
<< d['code/ex18.sh-session|pyg|l'] >>
\end{code}

这个技巧在 Windows 下的用法差不多,只不过你要执行的是 \verb|echo > somefile.txt|,
而且接下来命令行会提示让你输入每一行,当你输入一个空行(直接回车)时,这条命令也就结束了。

\begin{code}{Windows 练习 18}
<< d['code/ex18-win.sh-session|pyg|l'] >>
\end{code}

\section{知识点}

你创建了两个几乎一样的文件,只不过其中一个里边有“new”,而另一个是“old”。然后你
在这些文件中搜索不同的文字,你可以查找单词,不过如果你为要寻找的东西加上引号,
你也可以寻找整句内容。

\section{更多任务}

\begin{enumerate}
\item 使用引号寻找 "new file" 和 "old file" 以及 "This is"。
\item 在你创建的视频文件列表(或者任何文件列表)中查找你想要找到的某个文件。
\item Unix:你可以使用 \verb|-i| 让 \program{grep} 忽略大小写。执行 \verb|grep -i new *.txt| 试试看。
\end{enumerate}

