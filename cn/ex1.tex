\chapter{准备工作}

这本书将带领你做三件事情:

\begin{enumerate}
\item 在你的 shell,或者命令行(command line),或者终端(Terminal),或者 PowerShell 写一些东西。
\item 弄懂你刚写的东西。
\item 自己再多写一些东西。
\end{enumerate}

第一节练习中,你的目的是打开你的终端并确认它能正常工作,以便继续学习下去。

\section{任务}

准备好你的 Terminal,shell,PowerShell,设置好以便快速访问它们。

\subsection{Mac OSX}

对于 Mac OSX 你要做的如下:

\begin{enumerate}
\item 按住 COMMAND 键再敲击空格键。
\item 右上方会跳出蓝色的“搜索栏”。
\item 输入: terminal
\item 点击长得像一个黑盒子的 Terminal 应用程序。
\item 这样 Terminal 就打开了。
\item 你可以跑到 Dock 里面按着 CTRL 点击,在打开的菜单中选择 \verb|Options->Keep| 在 Dock 中。
\end{enumerate}

这样你就打开了 Terminal,而且 Dock 里也有了快速访问链接。

\subsection{Linux}

如果你已经在使用 Linux,那我就可以假设你已经知道如何找到 terminal 了。在你的 Window Manager
里搜寻名字像 "Shell" 或者 "Terminal" 的东西就可以了。

\subsection{Windows}

Windows 下我们将使用 PowerShell。人们以前用的一个叫 cmd.exe 的程序,不过和 PowerShell
比起来它的可用性差很多。如果你用的是 Windows 7 以上的系统,就照下面的做:

\begin{enumerate}
\item 点击 Windows 菜单。
\item 在搜索框中输入:powershell
\item 敲击回车键。
\end{enumerate}

如果你装的不是 Windows 7,那你应该认真考虑一下升级事宜。如果你实在不想升级,那就
试着从\href{http://www.microsoft.com/download/en/details.aspx?displaylang=en&id=16818}{微软的下载中心}
安装一下。这得靠你自己了,因为我没有装 Windows XP,写不出安装流程来,不过希望 XP
下的 PowerShell 的使用体验也是一样的吧。

\section{知识点}

你学会了如何打开你的 terminal,这是继续本书所必须的。

\begin{aside}{躲开所谓的黑客和他们的 zsh}

如果你有一个挺聪明而且懂 Linux 的朋友,再如果他介绍 bash 之外的 shell 给你,那你
应该忽略他们的建议。我教你的是 bash,就是这样。他们会说 zsh 会让你的 IQ 多长 30
个点,并且让你在股票市场上赚到腰缠万贯,别理他们就是了。你的目的只是学会足够的
技能,在你这个技能等级上,使用哪个 shell 其实不会影响到什么东西。

接下来的一个建议就是躲开 IRC 以及那些“黑客”常去的地方。他们会教你一些破坏你电脑
的命令并以此为乐。例如这条经典的 \verb|rm -rf /|,\emph{千万别输这条命令!}
离黑客远点,如果你需要帮助,就找你能信任的人,别去找网上乱七八糟的二货。
\end{aside}

\section{更多任务}

本节练习有一个很大的“更多任务”部分。其他的练习没这么多的额外任务要做,不过我
要让你通过记忆的方式向自己灌输足够的知识。跟着我走就行了,这会让你后面的学习
像丝一般顺畅。

\subsection{Linux/Mac OSX}

用索引卡片写下所有的命令,一张卡片写一条,正面写下命令,背面写下解释。每次看书就
学习一次,每次学个 15 分钟左右。

\begin{description}
\item[pwd] print working directory(打印工作目录)
\item[hostname] my computer's network name(电脑在网络中的名称)
\item[mkdir] make directory(创建路径)
\item[cd] change directory(更改路径)
\item[ls] list directory(列出路径下的内容)
\item[rmdir] remove directory(删除路径)
\item[pushd] push directory(推入路径)
\item[popd] pop directory(推出路径)
\item[cp] copy a file or directory(复制文件或路径)
\item[mv] move a file or directory(移动文件或路径)
\item[less] page through a file(逐页浏览文件)
\item[cat] print the whole file(打印输出整个文件)
\item[xargs] execute arguments(执行参数)
\item[find] find files(寻找文件)
\item[grep] find things inside files(在文件中查找内容)
\item[man] read a manual page(阅读手册)
\item[apropos] find what man page is appropriate(寻找恰当的手册页面)
\item[env] look at your environment(查看你的环境)
\item[echo] print some arguments(打印一些参数)
\item[export] export/set a new environment variable(导出/设定一个新的环境变量)
\item[exit] exit the shell(离开 shell)
\item[sudo] DANGER! become super user root DANGER!(成为超级用户或 root,危险命令!)
\item[chmod] change permission modifiers(修改文件许可权限)
\item[chown] change ownership(修改文件的所有者)
\end{description}

\subsection{Windows}

If you're using Windows then here's your list of commands:

\begin{description}
\item[pwd] print working directory(打印工作目录)
\item[hostname] my computer's network name(电脑在网络中的名称)
\item[mkdir] make directory(创建路径)
\item[cd] change directory(更改路径)
\item[ls] list directory(列出路径下的内容)
\item[rmdir] remove directory(删除路径)
\item[pushd] push directory(推入路径)
\item[popd] pop directory(推出路径)
\item[cp] copy a file or directory(复制文件或路径)
\item[robocopy] robust copy(更可靠的复制命令)
\item[mv] move a file or directory(移动文件或路径)
\item[more] page through a file(打印输出整个文件)
\item[type] print the whole file(打印输出整个文件)
\item[forfiles] run a command on lots of files(在一大堆文件上面运行一条命令)
\item[dir /r] find files(寻找文件)
\item[select-string] find things inside files(在文件中查找内容)
\item[help] read a manual page(阅读手册)
\item[helpctr] find what man page is appropriate(寻找恰当的手册页面)
\item[echo] print some arguments(打印一些参数)
\item[set] export/set a new environment variable(导出/设定一个新的环境变量)
\item[exit] exit the shell(离开 shell)
\item[runas] DANGER! become super user root DANGER!(成为超级用户或 root,危险命令!)
\item[attrib] change permission modifiers(修改文件许可权限)
\item[iCACLS] change ownership(修改文件的所有者)
\end{description}

不停地练习,直到你可以看到一条命令,然后能立即说出它的功能为止。然后反过来也能
说出实现每个功能所需的命令。通过这样的方式你可以为自己建立语汇,不过如果你觉得
烦了,也别强迫自己在上面花太多时间。