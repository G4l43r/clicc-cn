\chapter{流水式文件内容显示 (cat)}

你需要更多的准备工作,这个过程也会让你习惯这个工作流程:你在一个程序中创建文件,
然后通过命令行对其进行处理。使用练习 12 中的 \emph{文本编辑器} 创建一个叫 
 \file{ex13.txt} 的文件,不过这次将其直接保存到 \file{temp} 目录下。

\section{任务}

\begin{code}{Linux/Mac OSX 练习 13}
<< d['code/ex13.sh-session|pyg|l'] >>
\end{code}

\begin{code}{Windows 练习 13}
<< d['code/ex13-win.sh-session|pyg|l'] >>
\end{code}

记住,我写的 \verb|[displays file here]| 是表示我略掉了命令的输出,这样我就
不用把东西详尽地展示给你了。


\section{知识点}

我的诗怎么样?拿个诺贝尔奖没问题吧?不管怎样,你已经学到了第一个命令,而我只是
让你检查你的文件已经在那里了。然后你使用 \program{cat} 将文件内容显示到屏幕上。
这个命令会将整个文件一次输出到屏幕,不会分页也不会中间停顿。为了展示这一点,我让
你对 \file{ex12.txt} 执行 \program{cat} 命令,结果就是一下输出了文本中所有的行。 

\section{更多任务}

\begin{enumerate}
\item 再创建几个文本文件,然后用 \program{cat} 逐一打开。
\item Unix:试试 \verb|cat ex12.txt ex13.txt|,看看结果是怎样的。
\item Windows:试试 \verb|cat ex12.txt,ex13.txt| 看看结果。
\end{enumerate}

