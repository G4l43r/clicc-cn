\chapter{路径, 文件夹, 目录 (pwd)}

\section{任务}

接下来我要教你如何阅读我展示给你的这些终端会话。你不需要将所有的东西都键入终端,只是
其中的一部分内容而已:

\begin{enumerate} 
\item 你\emph{不需要}键入 \verb|$| (unix) 或者
    \verb|>| (Windows)。这个只是为了标记这是一个终端会话而已。
\item 你写完 \verb|$ or >| 后面的内容后需要敲击回车键。所以如果你看到了 \verb|$ pwd|,
那你需要输入 \verb|pwd| 再敲击一次回车键。
\item 你可以看到输出的内容前面也有一个 \verb|$| 或者 \verb|>| 提示。这些是输出
内容,你的输出内容和我的应该是一样的。
\end{enumerate}

让我们先试一个命令,看看你有没有弄明白:

\begin{code}{Linux/Mac OSX 练习 2}
<< d['code/ex2.sh-session|pyg|l'] >>
\end{code}

\begin{code}{Windows 练习 2}
<< d['code/ex2-win.sh-session|pyg|l'] >>
\end{code}

\begin{aside}{命令提示符之 Windows vs Unix }

为了节约空间同时让你集中精力到重要的命令细节上面,本书将把命令行一开始的部分
(例如上面的 \verb|PS C:\Users\zed| )省略掉,只留下一个小小的 \verb|>| 部分。
这意味着你的命令行和这里看到的会有一点不同,不过这个是正常的,你无须操心。

记住从现在开始,我只通过 \verb|>| 来告诉你这是一个命令行提示。

对于 Unix 命令行提示也一样,不过 Unix 有点不一样,人们习惯使用 \verb|$| 来表示
命令提示符。

\end{aside}

\section{知识点}

你的命令行和我的看上去不一样,你的可能在 \verb|$| 前面显示了你的用户名以及计算机
名。Windows 下看上去也会不一样,不过关键的基本格式都是这样的:

\begin{enumerate}
\item 有一个命令提示符。
\item 你输入一条命令,例如这里的 pwd。
\item 你得到一些输出。
\item 重复上述步骤。
\end{enumerate}

你正好还学会了 \program{pwd} 的功能,它的意思是“打印工作目录(print working
directory)”。目录是什么东西?就是文件夹而已。文件夹和目录是一样的东西,这两个
词可以交替使用。当你打开文件浏览器,通过图形界面寻找文件,那你就是在访问文件夹。
这些文件夹和我们后面要用到的目录完全是同一回事。

\section{更多任务}

\begin{enumerate}
\item 输入 \program{pwd} 20 次,每次输入都念一遍“print working directory”。
\item 记下命令行显示的路径,用你的文件浏览器找到这个位置。
\item 我不是开玩笑。写 20 遍,并且朗读出来。别抱怨了,照我说的做。
\end{enumerate}


