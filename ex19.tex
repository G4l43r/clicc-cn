\chapter{Getting Command Help (man, HELP)}

\section{Do This}

\begin{code}{Linux/OSX Exercise 19}
\begin{Verbatim}
<< d['code/ex19.sh-session'] >>
\end{Verbatim}
\end{code}

\begin{code}{Windows Exercise 19}
\begin{Verbatim}
<< d['code/ex19-win.sh-session'] >>
\end{Verbatim}
\end{code}

\section{You Learned This}

You can use the \program{man} command on Unix, and the \program{help} command
in Windows to find information about commands.  I've been playing a very dirty
trick on you this whole time.  I could have just told you to do this and read
about each command rather than memorize what they do.  But, if I did this you'd
be lost because you wouldn't know the basic commands you have to know, or how
directories work, or what a wildcard was, or other things.

Hopefully you forgive me for not telling you about this awesome tool until now,
but now when you forget what a command does, just use the help.

\section{Do More}

\begin{enumerate}
\item Use \program{man} or \program{help} to look at every one of the commands you have in your list to memorize.
\end{enumerate}

