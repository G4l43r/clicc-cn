\chapter{Pipes And Redirection}

\section{Do This}

\begin{code}{Linux/Mac OSX Exercise 15}
<< d['code/ex15.sh-session|pyg|l'] >>
\end{code}

\begin{code}{Windows Exercise 15}
<< d['code/ex15-win.sh-session|pyg|l'] >>
\end{code}

\section{You Learned This}

Now we get to the cool part of the command line: redirection.  The concept is
that you can take a command and you can change where its input and output goes. 
You use the \verb|<| (less-than), \verb|>| (greater-than), and \verb,|, (pipe) 
symbols to do this.  Here's a breakdown:

\begin{description}
\item[$|$] The \verb,|, takes the output from the command on the left, and "pipes" it to the command on the right.  In line 1 you see me do that.
\item[$<$] The \verb|<| will take and send the input from the file on the right to the program on the left.  You see me do that in line 2. \emph{This does not work in PowerShell}.
\item[$>$] The \verb|>| takes the output of the command on the left, then writes it
    to the file on the right.  You see me do that on line 9. 
\item[$>>$] The \verb|>>| takes the output of the command on the left, then \emph{appends} it
    to the file on the right.  You see me do that on line 9.
\end{description}

There are a few more symbols, but we'll start with just these for now.

\section{Do More}

\begin{enumerate}
\item Create some more index cards for memorizing these three symbols.  Write the symbol on one side, then what it does on the other side.  Drill these just like the other commands.
\end{enumerate}

