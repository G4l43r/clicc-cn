\chapter{The Setup}

In this book you will be instructed to do three things:

\begin{enumerate}
\item Do some things in your shell (command line, terminal, cmd.exe).
\item Learn about what you just did.
\item Do more on your own.
\end{enumerate}

For this exercise you'll be expected to get your Terminal open and working so that you can do
the rest of the book.

\section{Do This}

Get your terminal, shell, cmd.exe working so you can access it quickly and know that it works.

\subsection{Mac OSX}

For Mac OSX you'll need to do this:

\begin{enumerate}
\item Hold down COMMAND and hit the spacebar.
\item In the top right the blue "search bar" will pop up.
\item Type:  terminal
\item Click on the Terminal application, that looks kind of like a black box.
\item This will open Terminal.
\item You can now go to your Dock and CTRL-click to pull up the menu, then select Options->Keep In Dock.
\end{enumerate}

Now you have your Terminal open and it's in your Dock so you can get to it.

\subsection{Linux}

I'm assuming that ifyou have Linux then you already know how to get at your terminal.  Look through the
menu for your window manager for anything named "Shell" or "Terminal".

\subsection{Windows}

On windows we're going to use PowerShell. People used to work with this thing called cmd.exe, but it's not nearly as usable as PowerShell. If you have Windows 7 or later do this:

\begin{enumerate}
\item Click Start.
\item In "Search programs and files" put: powershell
\item Hit Enter.
\end{enumerate}

If you don't have Windows 7, you should *seriously* consider upgrading. If you
still insist on not upgrading then you can try installing it
\href{http://www.microsoft.com/download/en/details.aspx?displaylang=en&id=16818}{from
the download center}. You are on you're own though since I don't have Windows
XP, but hopefully the PowerShell experience is the same. 

\section{You Learned This}

You learned how to get your terminal open so you can do the rest of this book.

\begin{aside}{Avoid The Hackers and Their zsh}
If you have that really smart friend who already knows Linux, ignore them when they tell you to use something
other than bash.  I'm teaching you bash.  That's it.  They will claim that zsh will give you 30 more IQ points
and win you millions in the stock market.  Ignore them.  Your goal is to get capable enough and at this level
it doesn't matter what shell you use.

The next warning is stay off IRC or other places where "hackers" hang out.  They think it's funny to hand you
commands that can destroy your computer.  The command \verb|rm -rf /| is a classic that you \emph{must never type}.
Just avoid them.  If you need help, make sure you get it from someone you trust not from random idiots on the
internet.
\end{aside}

\section{Do More}

This exercise has a large "do more" part.  The other exercises are not as involved as this one, but
I'm having you prime your brain for the rest of the book by doing some memorization.  Just trust me,
this will make things silky smooth later on.

\subsection{Linux/OSX}

Take this list of commands and create index cards with the names on the left
on one side, and the definitions on the other side.  Drill them every day while
you do this book for just 15 minutes or so.

\begin{description}
\item[pwd] print working directory
\item[whoami] who am i
\item[mkdir] make directory
\item[cd] change directory
\item[ls] list directory
\item[rmdir] remove directory
\item[pushd] push directory
\item[popd] pop directory
\item[cp] copy a file or directory
\item[mv] move a file or directory
\item[less] page through a file
\item[cat] print the whole file
\item[xargs] execute arguments
\item[find] find files
\item[grep] find things inside files
\item[man] read a manual page
\item[apropos] find what man page is appropriate
\item[env] look at your environment
\item[echo] print some arguments
\item[export] export/set a new environment variable
\item[exit] exit the shell
\item[sudo] DANGER! become super user root DANGER!
\item[chmod] change permission modifiers
\item[chown] change ownership
\end{description}

\subsection{Windows}

If you're using Windows then here's your list of commands:

\begin{description}
\item[pwd] print working directory
\item[whoami] who am i
\item[mkdir] make directory
\item[cd] change directory
\item[ls] list directory
\item[rmdir] remove directory
\item[pushd] push directory
\item[popd] pop directory
\item[cp] copy a file or directory
\item[robocopy] robust copy
\item[mv] move a file or directory
\item[more] page through a file
\item[type] print the whole file
\item[forfiles] run a command on lots of files
\item[dir /r] find files
\item[select-string] find things inside files
\item[help] read a manual page
\item[helpctr] find what man page is appropriate
\item[echo] print some arguments
\item[set] export/set a new environment variable
\item[exit] exit the shell
\item[runas] DANGER! become super user root DANGER!
\item[attrib] change permission modifiers
\item[iCACLS] change ownership
\end{description}

Drill, drill, drill! Drill until you can say these phrases right away when you
see that word.  Then drill the inverse, so that you read the phrase and know
what command will do that.  You're building your vocabulary by doing this, but
don't spend so much time you go nuts and get bored.

\subsection{Windows}

Just like above, here's the same list but using the Windows versions:

TODO:  Need the windows version.
