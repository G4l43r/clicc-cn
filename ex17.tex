\chapter{Finding Files (find, DIR -R)}

\section{Do This}

This exercise is going to combine three concepts into one single command.  I'm
going to show you how to find all the text files and page through them.

\begin{code}{Linux/OSX Exercise 17}
<< d['code/ex17.sh-session|pyg|l'] >>
\end{code}

\begin{code}{Windows Exercise 17}
<< d['code/ex17-win.sh-session|pyg|l'] >>
\end{code}

\section{You Learned This}

You just learned how to run the \program{find} command to search for all files
ending in \file{.txt} and then look at the results with \program{less} (\program{more} on Windows).  Here's a breakdown of exactly what this does:

\begin{enumerate}
\item First I go to the \file{temp} directory.
\item Then I just do a plain find from there since there's not many \file{.txt}
files.  How find works is you write in a kind of sentence: "Hey \program{find},
    start here (.) then find files named "*.txt" and print them".  Thinking
    about commands as if you're telling the computer to do something is a good
    way to remember it.
\item Next I go up one directory and then I do the same command, but this time I pipe the output to \program{less} (\program{more} on Windows).  This command could run for a while, so be patient.
\item I then do this again, but this one could take a really long time, so feel free to just hit CTRL-c to abort it.
\end{enumerate}


\section{Do More}

\begin{enumerate}
\item For Unix, get your \program{find} index card and add this to the description side: "find STARTDIR -name WILDCARD -print".  Next time you drill make sure you can say that phrase so you remember how find is formatted.
\item For Windows do the same but write: TODO.
\item You can put any directory where the \verb|.| (dot) is.  Try another directory to start your search there.
\item Look for all the video files on your computer starting at the home drive and use the \verb|>| to save the list to a file.
\end{enumerate}

